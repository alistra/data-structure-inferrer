\documentclass[11pt]{article}
\usepackage[utf8]{inputenc}
\usepackage[colorinlistoftodos, textwidth=4cm, shadow]{todonotes}
\usepackage{listings}
\lstset{language=C}


%\usepackage[]{polski}

\newcommand{\lecture}[4]{\handout{#1}{#2}{#3}{#4}{#1}}
\newcommand{\high}{\mathrm{high}}
\newcommand{\low}{\mathrm{low}}

\newtheorem{theorem}{Twierdzenie}
\newtheorem{corollary}[theorem]{Wniosek}
\newtheorem{lemma}[theorem]{Lemat}
\newtheorem{observation}[theorem]{Obserwacja}
\newtheorem{definition}[theorem]{Definicja}
\newtheorem{fact}[theorem]{Fakt}
\newtheorem{assumption}[theorem]{Założenie}

% 1-inch margins, from fullpage.sty by H.Partl, Version 2, Dec. 15, 1988.
\topmargin 0pt
\advance \topmargin by -\headheight
\advance \topmargin by -\headsep
\textheight 8.9in
\oddsidemargin 0pt
\evensidemargin \oddsidemargin
\marginparwidth 0.5in
\textwidth 6.5in

\parindent 0in
\parskip 1.5ex
%\renewcommand{\baselinestretch}{1.25}

\title{\emph{dsinf}: Source Based Data Structure Inference}
\author{Aleksander Balicki}
\date{\today}

\begin{document}

\maketitle

\begin{abstract}
	When you need to store data, most of popular languages today already have
	libraries with all the important data structures implemented.
	Programmers just have to be taught when to use them.
	Some of the cases of choosing the right data structure look sufficiently easy,
	so a computer could do it automatically. This work describes the \emph{dsinf} project,
	a framework for inferring the best data structure matching your task,
	based on the program's source code in C language.\missingfigure{
	result, conclusion, http://www.ece.cmu.edu/~koopman/essays/abstract.html}
\end{abstract}

\pagebreak

\tableofcontents

\vfill

\section{Introduction}
	\missingfigure{Introduction here}

\section{Data structure inference}

	\subsection{Comparison of the complexities}
		When trying to define some kind of ordering on data structures
		- so we can decide which is best at the moment - we encounter a problem.
		We want to compare the asymptotical complexities of operations on data structures.
		We can't do it for the general case, because it's too complicated,
		because an arbitrary function can describe the complexity of an operation.

		Asymptotical complexity of an operation we store as a pair of type:
		\begin{eqnarray}
			AsymptoticalComplexity = Int \times Int,
		\end{eqnarray}
		where
		\begin{eqnarray}
			(k, \; l) \; means \; O(n^k \log^l{ n}).
		\end{eqnarray}
		The reason to choose such a type is that it's easier to compare than the
		general case (we can do a lexicographical comparison of the two numbers)
		and it distincts most of the data structure operation complexities.

		Sometimes we have to use some qualified complexities:
		\begin{eqnarray}
			ComplexityType = \{ Normal, \; Amortized, \; Amortized \;Expected, \; Expected \}
		\end{eqnarray}

		The overall complexity can be seen as a type:
		\begin{eqnarray}
			Complexity = AsymptoticalComplexity \times ComplexityType
		\end{eqnarray}
		Here we can also use a lexicographical comparison, but we have to say that
		\begin{eqnarray}
			Amortized < Normal,\\
			Expected < Amortized,\\
			Amortized \; Expected < Expected
		\end{eqnarray}
		and that $<$ is transitive.

		We also always choose the smallest asymptotic-complexity-wise complexity.
		For example, we have a search operation on a splay tree. It's $O(n)$, but $O(\log n)$ amortized,
		so it's represented as $((0,1),Amortized)$.
	\subsection{Choosing the best data structure}
		We define a set $DataStructureOperations$. We can further extend this set, but for now assume that
		\begin{eqnarray}
		  	DataStructureOperations = \{Insert, \; Update, \; Delete, \; FindMax,\; DeleteMax, \; \dots\}.
		\end{eqnarray}
		Each of the $DataStructureOperations$ elements symbolizes an operation you can accomplish on a data structure.

		The type
		\begin{eqnarray}
			DataStructure = DataStructureOperations \rightarrow Complexity
		\end{eqnarray}
		represents a data structure and all of the operations implemented for it,
		with their complexities, as a partial function from DataStructureOperations to Complexities.

		When trying to find the best suited data structure for a given program $P$,
		we look for data structure uses in $P$. Let $DSU(P) :: P(DataStructureOperations)$ be the set
		of data structure operations, that are used somewhere in the source code of $P$.

		We define a parametrized comparison operator for data structures $<_{DSU(P)}$ defined as:
		\begin{center}
			\begin{equation}
				d_1 <_{DSU(P)} d_2
			\end{equation}
				$\Updownarrow$
			\begin{equation}
				|\{(o, c_1) \in d_1 | o \in DSU(P) \wedge (o,c_2) \in d_2 \wedge c_1 < c_2 \}| <
				0.5 * \lfloor |DSU(P)| \rfloor
			\end{equation}
		\end{center}
		If a data structure implements more operations 'faster' than we choose that structure over the other one.


		If we fix P, we have a preorder on data structures induced by $<_{DSU(P)}$
		and we can sort those data structures using this order. The maximum element is the best data structure for the task.
	\subsection{Collecting the program data}

\pagebreak

\section{Extensions of the idea}
	\subsection{Second extremal element}
		If we want to find the maximal element in a heap, we just look it up in $O(1)$, that's what heaps are for.
		If we want to find the minimal element we can modify the heap, for it to use an order,
		which would allow us to lookup the minimal element in $O(1)$.
		What happens if we want to find the max and the min element in the duration of one program?
		How to modify our framework to handle this kind of situations?
		  	$$DataStructureOperations = \{\dots, \; FindFirstExtremalElement,\\$$
			$$	DeleteFirstExtremalElement,\\$$
			$$	FindSecondExtremalElement,\\$$
			$$	DeleteSecondExtremalElement, \; \dots\}.$$
		Now we can add two complexity costs to the data structure definition. We can always reverse the order,
		so the cheaper one can be used primarily, and the more expensive one in situations when we need
		both types of extremal elements.
	\subsection{Detecting importance of an operation}
		Detecting the importance of a single data structure operation is an important problem, mainly because it's undecidable.
		This program shows that the problem of compile-time deciding on the best data structure is impossible to solve.
		\lstinputlisting{thesis-pics/undecidable.c}
		In the above example, the best data structure is depending on the user input, which is not known at compile time.

		Also code that is not totally dependent on user input can cause problems to analyze.
		\lstinputlisting{thesis-pics/weighted-instructions.c}
		Here we see a very costly instruction used only once, and a few instructions run in a loop for a few million times.
		To anyone that knows program complexities it's obvious that this one heavy instruction
		doesn't affect the execution time of the whole program, yet the framework at the current state
		treats those instructions equally.

		\subsubsection{Code pragmas}
			A possible soultion to this problem is to let the programmer add code pragmas to his source code,
			so he decides how important an instruction is in relation to other instructions and then the framework
			makes use of those values in choosing the data structure.
			\lstinputlisting{thesis-pics/code-pragmas.c}
			In the above example programmer can add a weights to the operations,
			assigning very low values for statements that are used rarely or for debug purposes,
			and very high values for crucial parts of the program.

			There would be an API change needed.
			\lstinputlisting{thesis-pics/code-pragmas-api-change.c}
			This isn't the perfect solution, because we still need the programmer to judge
			which operations should have high weights, but it's nice when a programmer
			wants to use it for debug purposes or otherwise tinker with it.

		\subsubsection{Choosing the best data structure with weigths}

		\subsubsection{Profile-guided optimization}
			Profile-guided optimization is an optimization method in compilers. -definition-.
			Here we can check how many times an operation is executed on our test-data
			and then choose the recommended structure accordingly.
		\subsubsection{Transforming datastructures on-line}
	\subsection{Generic data structure modifications}

		max elem cache/linked leaves
	\subsection{Different element types}
		Currently the framework works only for integer elements. We can extend it to each type that has a compare function,
		because there's no difference if the types are numerical or not.
		If we analyzed haskell, we would use types that are in the Ord class,
		if we analyzed C, we would ask the programmer to write an acompanying cmp function to the type and pass the pointer to the function that creates the data structure.
	\subsection{Linked data structures}
		If we wanted to keep records (structs) in our data structure and find an element by one field and some other time by some other field, we would have to do this:
	\subsection{Upper bound on the element count}
		so we can choose between malloc and static allocation
	\subsection{Outer-world input}
		detecting scanf and sockets and so on
	\subsection{Minimal element count treshold}
		It's worth noticing that we compare only the asymptotical complexity of data structures. Some awfully complicated structures can have good asymptotical results, but the constant is quite high. We can avoid this problem by setting a treshold for each structure, what is the smallest number of elements to use this data structure.

		Another problem arises, how to know at compile time, how many elements a data structure will have at runtime. We can ask the user to explicitely specify the number during compilation or we can try to detect how big the declared data is.
\section{Program}
	\subsection{Recommendation mode}
		prints recommendations
	\subsection{Advice mode}
		prints advice
	\subsection{Compile mode}
		linkes appropriate lib

\end{document}
