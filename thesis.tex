\documentclass[11pt]{article}
\usepackage[utf8]{inputenc}
%\usepackage[]{polski}

\newcommand{\lecture}[4]{\handout{#1}{#2}{#3}{#4}{#1}}
\newcommand{\high}{\mathrm{high}}
\newcommand{\low}{\mathrm{low}}

\newtheorem{theorem}{Twierdzenie}
\newtheorem{corollary}[theorem]{Wniosek}
\newtheorem{lemma}[theorem]{Lemat}
\newtheorem{observation}[theorem]{Obserwacja}
\newtheorem{definition}[theorem]{Definicja}
\newtheorem{fact}[theorem]{Fakt}
\newtheorem{assumption}[theorem]{Założenie}

% 1-inch margins, from fullpage.sty by H.Partl, Version 2, Dec. 15, 1988.
\topmargin 0pt
\advance \topmargin by -\headheight
\advance \topmargin by -\headsep
\textheight 8.9in
\oddsidemargin 0pt
\evensidemargin \oddsidemargin
\marginparwidth 0.5in
\textwidth 6.5in

\parindent 0in
\parskip 1.5ex
%\renewcommand{\baselinestretch}{1.25}

\title{Data structure inference based on source code}
\author{Aleksander Balicki}
\date{\today}

\begin{document}

\maketitle

\begin{abstract}
\end{abstract}

\section{Introduction}
\section{Data structure inference}
\subsection{Data structure modifications}
max elem cache
\subsection{Linked data structures}
keeping records
\subsection{Transforming datastructures on-line}
what it said
\subsection{Upper bound on the element count}
so we can choose between malloc and static allocation
\section{Program}
\subsection{Recommendation mode}
prints recommendations 
\subsection{Advice mode}
prints advice
\subsection{Compile mode}
linkes appropriate lib

\end{document}

