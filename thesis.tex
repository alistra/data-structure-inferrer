\documentclass[a4paper,11pt]{article}
\usepackage[utf8]{inputenc}
\usepackage[colorinlistoftodos, textwidth=2cm, shadow]{todonotes}
\usepackage{listings}
\usepackage[colorlinks=true,linkcolor=blue]{hyperref}
\usepackage[section]{placeins}
\lstset{language=C}

%\usepackage[]{polski}

% 1-inch margins, from fullpage.sty by H.Partl, Version 2, Dec. 15, 1988.
\topmargin 0pt
\advance \topmargin by -\headheight
\advance \topmargin by -\headsep
\textheight 8.9in
\oddsidemargin 0pt
\evensidemargin \oddsidemargin
\marginparwidth 0.5in
\textwidth 6.5in
\parindent 0in
\parskip 1.5ex
%\renewcommand{\baselinestretch}{1.25}

\title{\emph{dsinf}: Source Based Data Structure Inference}
\author{Aleksander Balicki}
\date{\today}

\begin{document}

\vfill

\maketitle

\begin{abstract}

	When you need to store data, \todo{Connect the parts of the sentence} most of popular languages today already
	have libraries with all the important data structures implemented.  Programmers just have to be taught when to
	use them.  Some of the cases of choosing the right data structure look sufficiently easy, so a computer could do
	it automatically. This work describes the \emph{dsinf} project, a framework for inferring the best data
	structure matching your task, based on the program's source code in C language. \missingfigure{ result,
	conclusion, http://www.ece.cmu.edu/~koopman/essays/abstract.html}

\end{abstract}

\pagebreak

\tableofcontents

\vfill

\section{Introduction}
	In computer science, a data structure is a particular way of storing and organizing data in a computer so that
	it can be used efficiently.\todo{wikipedia}

	The interface of a data stucture is a set of operation that you can perform on a data structure and some
	semantics of those functions.

	The implementation of a data structure is a realization of some data structure interface in some programming
	language, that is consistent with the semantics.


	We can distinguish a lot of different types of data stuctures:
	\begin{itemize}
		\item Map/Associative array/Dictionary is a data structure for storing $(key,\;value)$ pairs, it allows
			insertion, deletion, modification and lookup
		\item Multimap is the same as Map, but with the possibility of storing multiple values for a one key
		\item List is a data structure for storing ordered sets of values
		\item Set is a data structure for storing unordered sets of values
		\item Multiset is the same as Set, but with the possibility of storing multiple objects that are equal
			under some equivalence relation
		\item Queue is a data structure implementing operations for enqueuing and dequeuing elements
		\item Deque is a queue, where you can add and remove elements from both ends as opposing to one
		\item Stack is a data structure implementing operations for putting something on the top of the stack or
			removing the top element from it
		\item Priority queue is like a regular queue, but each element has a priority associated with it, which
			is used for ordered dequeuing of the elements
		\item String is a data structure for storing text
		\item Tree is a data structure, to hold some tree-like hierarchical structure
		\item Graph is a data structure, to store elements and a relation on those elements
		\item Others, which usually are a mix of those previous ones
	\end{itemize}

	\missingfigure{Introduction here}
	\subsection{Comparing a database} \label{sub:database}

\section{Data structure inference}

	\subsection{Comparison of the complexities}

		When trying to define some kind of ordering on data structures - so we can decide which is best at the
		moment - we encounter a problem.  We want to compare the asymptotical complexities of operations on data
		structures.  We can't do it for the general case, because it's too complicated, because an arbitrary
		function can describe the complexity of an operation.

		Asymptotical complexity of an operation we store as a pair of type:

		\begin{eqnarray}
			AsymptoticalComplexity = Int \times Int,
		\end{eqnarray}

		where

		\begin{eqnarray}
			(k, \; l) \; means \; O(n^k \log^l{ n}).
		\end{eqnarray}

		The reason to choose such a type is that it's easier to compare than the general case (we can do a
		lexicographical comparison of the two numbers) and it distinguishes most of the data structure operation
		complexities.

		Sometimes we have to use some qualified complexities:

		\begin{eqnarray}
			ComplexityType = \{ Normal, \; Amortized, \; Amortized \;Expected, \; Expected \}
		\end{eqnarray}

		The overall complexity can be seen as a type:

		\begin{eqnarray}
			Complexity = AsymptoticalComplexity \times ComplexityType
		\end{eqnarray}

		Here we can also use a lexicographical comparison, but we have to say that

		\begin{eqnarray}
			Amortized < Normal,\\
			Expected < Amortized,\\
			Amortized \; Expected < Expected
		\end{eqnarray}

		and that $<$ is transitive.

		We also always choose the smallest asymptotic-complexity-wise complexity.  For example, we have a search
		operation on a splay tree. It's $O(n)$, but $O(\log n)$ amortized, so it's represented as
		$((0,1),Amortized)$.

	\subsection{Choosing the best data structure}

		We define a set $DataStructureOperations$. We can further extend this set, but for now assume that

		\begin{eqnarray}
			DataStructureOperations = \{Insert, \; Update, \; Delete, \; FindMax,\; DeleteMax, \; \dots\}.
		\end{eqnarray}

		Each of the $DataStructureOperations$ elements symbolizes an operation you can accomplish on a data
		structure.

		The type

		\begin{equation}\label{data-structure-type}
			DataStructure = DataStructureOperations \rightarrow Complexity
		\end{equation}

		represents a data structure and all of the operations implemented for it, with their complexities, as a
		partial function from DataStructureOperations to Complexities.

		When trying to find the best suited data structure for a given program $P$, we look for data structure
		uses in $P$. Let \todo{DSU is a ds, should be [ds]}

		\begin{equation}\label{dsu-type}
			DSU(P) :: P(DataStructureOperations)
		\end{equation}

		be the set of data structure operations, that are used somewhere in the source code of $P$.

		We define a parametrized comparison operator for data structures $<_{DSU(P)}$ defined as:

		\begin{center}

			\begin{equation}
				d_1 <_{DSU(P)} d_2
			\end{equation}

			$\Updownarrow$

			\begin{equation} \label{data-structure-order}
				|\{(o, c_1) \in d_1 | o \in DSU(P) \wedge (o,c_2) \in d_2 \wedge c_1 < c_2 \}| < 0.5 *
				\lfloor |DSU(P)| \rfloor
			\end{equation}

		\end{center} If a data structure implements more operations 'faster' than we choose that structure over
		the other one.

		If we fix P, we have a preorder on data structures induced by $<_{DSU(P)}$ and we can sort those data
		structures using this order. The maximum element is the best data structure for the task.

	\subsection{Collecting the program data} \label{dsu-definition}

		In \autoref{dsu-type} the type of the $DSU()$ operation was mentioned, this section shows, how $DSU()$ is
		defined.

		$DSOPS(P)$ is a set sum of:

		\begin{enumerate}

			\item $\{(g, o)\; |\; g \texttt{ is a global data structure variable in $P$ } \wedge \\
				o \texttt{ is an operation performed on $g$, somewhere in $P$} \}$

			\item $\{(d, o)\; |\; d \texttt{ is declared somewhere in a body of a function $f$ in $P$ } \wedge \\
				o \texttt{ is an operation performed on $d$, somewhere in $f$ } \}$

			\item $\{(p, o)\; |\; p \texttt{ is a formal parameter of a function $f$ in $P$ } \wedge \\
				o \texttt{ is an operation performed on $p$, somewhere in $f$ } \}$

		\end{enumerate}

		\subsubsection{C API}
			split in to dselem and ds

		\subsubsection{Analysis}

			graphs in graphviz

\pagebreak

\section{Extensions of the idea}

	\subsection{Second extremal element}

		If we want to find the maximal element in a heap, we just look it up in $O(1)$, that's what heaps are
		for.  If we want to find the minimal element we can modify the heap, for it to use an order, which would
		allow us to lookup the minimal element in $O(1)$.  What happens if we want to find the max and the min
		element in the duration of one program?  How to modify our framework to handle this kind of situations?

		\begin{eqnarray*}
			DataStructureOperations = \{\dots, \; FindFirstExtremalElement,\\
			DeleteFirstExtremalElement,\\
			FindSecondExtremalElement,\\
			DeleteSecondExtremalElement\}.
		\end{eqnarray*}

		Now we can add two complexity costs to the data structure definition. We can always reverse the order,
		so the cheaper one can be used primarily, and the more expensive one in situations when we need both
		types of extremal elements.

	\subsection{Importance of operations}

		Detecting the importance of a single data structure operation is an important problem, because a lot of
		programs will fall into a class where compilation time data structure selection is undecidable. This
		program shows that the problem of compile-time deciding on the best data structure is impossible to
		solve.

		\begin{figure}
			\lstinputlisting{thesis-pics/undecidable.c}

			\caption{An example of the problem of choosing the fastest data structure for a program being
			undecidable, user interaction}

			\label{fig:undecidable}
		\end{figure}

		In the \autoref{fig:undecidable}, the best data structure is depending on the user input, which is not known at
		compile time. If we analyze it as is, it will decide basing on the information, that every operation is
		used, so the final data structure won't be specific to the task, but just average at everything.

		Also code that is not totally dependent on user input can cause problems to analyze.

		\begin{figure}
			\lstinputlisting{thesis-pics/weighted-instructions.c}

			\caption{An example of the problem of choosing the fastest data structure for a program being
			undecidable, no user interaction}

			\label{fig:undecidable-no-user}
		\end{figure}

		In the \autoref{fig:undecidable-no-user} we see a very costly instruction used only once
		($delete\_max\_d$), and a few instructions ($insert$, $search$) run in a loop for a few million times.
		To anyone that knows program complexities it's obvious that this one heavy instruction doesn't affect
		the execution time of the whole program, yet the framework at the current state treats those
		instructions equally and will probably choose something like Red-Black Trees, to minimize the time of
		the $delete\_max\_d$, instead of ignoring its cost and using a very fast Hashtable.

		\subsubsection{Code pragmas}

			A possible solution to this problem is to let the programmer add code pragmas to his source
			code, so he decides how important an instruction is in relation to other instructions and then
			the framework makes use of those values in choosing the data structure.

			\begin{figure}
				\lstinputlisting{thesis-pics/code-pragmas.c}

				\caption{Source code with pragmas, that state the importance of the operation}

				\label{fig:code-pragmas}
			\end{figure}

			In the above example programmer can add a weights to the operations, assigning very low values
			for statements that are used rarely or for debug purposes, and very high values for crucial
			parts of the program.

			There would be an API change needed.

			\begin{figure}
				\lstinputlisting{thesis-pics/code-pragmas-api-change.c}

				\caption{dsinf API, using additional arguments for operation importance}

				\label{fig:code-pragmas-api}
			\end{figure}

			This isn't the perfect solution, because we still need the programmer to judge which operations
			should have high weights, but it's nice when a programmer wants to use it for debug purposes or
			otherwise tinker with it.

		\subsubsection{Choosing the best data structure with weights}

			We need to change the algorithm for choosing the best data structure, for it to handle weights
			to the operations in the source code. We change the type of the $DSU()$ operation from
			\autoref{dsu-type} to:

			\begin{equation}
				DSU_w(P) :: DataStructureOperations -> Int
			\end{equation}

			This change introduces weights for operations in program $P$. We can use a partial function, or
			just use $0$ as a value for operations that do not occur in the program. We always want to get
			the highest weight from all the weighted operation instances in the program, so the $DSU()$
			definition also needs a change from \autoref{dsu-definition} \todo{do the equation, when you have
			the first definition}:

			\begin{equation}
				DSU_w(P) =
			\end{equation}

			We change the definition of \autoref{data-structure-order} to:

			\begin{equation}
			\end{equation}

		\subsubsection{Profile-guided optimization}

			Profile-guided optimization (PGO) is a compiler optimization technique in computer programming
			to improve program runtime performance.  In contrast to traditional optimization techniques that
			solely use the source code, PGO uses the results of test runs of the instrumented program to
			optimize the final generated code \todo{reference wikipedia}.

			If the user has some test data, he can run against his program, we can take advantage of that.
			First we choose the best data structure with an unmodified method and link some library to the
			executable. Of course this won't be the best data structure possible. Then user can run the test
			suite with code coverage option, like \emph{gcov} in GCC, turned on in the compiler.

			\begin{figure}
				\lstinputlisting{thesis-pics/gcov.c}

				\centering \line(1,0){450}

				\lstinputlisting{thesis-pics/gcov.c.gcov}

				\caption{Example of a source code file and the .gcov file, generated by running the
				compiled program}

				\label{fig:gcov}
			\end{figure}


			Then the user can pass the \emph{gcov} generated files and the source code to the framework
			again. The framework can extract line hits from the \emph{gcov} files on the data structure
			operations and set weights on the operations according to the extracted data and then use the
			choosing algorithm for operations with weights.

			It's a better solution than letting the user set the weights himself, but still, the data the
			inference is based upon, comes from tests and there's no guarantee the real world data will
			match the test data.

		\subsubsection{Transforming data structures on-line} \label{sec:transforming-on-line}

	\subsection{Generic data structure modifications}

		When searching for the most apt data structure, we need to have some kind of data structure database,
		where we keep all the structures' metadata and implementations for the framework to use. Ideally there
		would be a lot of different data structures there.

		When implementing a data structure, one could easily modify the implementation to maybe match some rare
		specific task, that is not needed in most cases. It would be wasteful to keep a copy of the data
		structure for each small combination of those modifications, if we want to return a data structure
		tailored perfectly to the task. Of course some of the modifications are very data structure specific, so
		a generalization isn't possible, but for some cases, we can extract a piece of code, like a wrapper for
		a data structure, that modifies its behavior in some specific way.\todo{reference okasaki}

		\subsubsection{Extremal element cache}

			When we want to be able to lookup extremal elements in the data structure. We don't really have
			to know the implementation specifics of the data stucture, we only need to intercept the calls
			which insert and delete elements. When we insert an element, we compare it to a cache variable,
			which keeps the biggest element (if our extremal element is the maximal element) known to be in
			the data structure.

			But this road also has some drawbacks, if we delete an element, and it occurs that the deleted
			element is the current maximum, we have to find the new maximal element from the rest of
			elements, which can be asymptoticaly more costly than the delete itself.

			\missingfigure{table of cost changes}

			\begin{figure}
				\lstinputlisting{thesis-pics/elem-cache.c}

				\caption{An example implementation of the extremal element cache data structure wrapper}

				\label{fig:elem-cache}
			\end{figure}

			We notice that a lot of data strucutures already implement this pattern \todo{which ones, veb?}.

		\subsubsection{Linked elements}

			When we have a structure that keeps some data, to find a precedessor or a successor, it usually
			takes $O(n)$ or $O(lg n)$ operations.  To help this we can create an overlay with a doubly
			linked list on the elements, so our precedessor/successor lookups are $O(1)$. Drawback here is
			that the cost of insert is increased by searching the successor to link it appropriately.

			\begin{figure}
				\lstinputlisting{thesis-pics/linked-elements.c}

				\caption{An example implementation of the linked elements data structure wrapper}

				\label{fig:linked-cache}
			\end{figure}

	\subsection{Different element types}

		Currently the framework works only for integer elements. We could extend it to every primitive type in
		C, but it would require some changes.  Some data structures require the type to be comparable, which
		wouldn't be a big problem, because there is a comparison semantics defined on those types. There's also
		a hash function needed, because some data structures, like a hashtable, need to compute hash values to
		work. We can just use the binary representation of other types, and use it as an integer for the hash
		function. This whole modification doesn't need any user interaction.

		There's a bigger problem with composite types. Comparing pointers isn't obvious. You may want to compare
		addresses or the contents under that address, depending if you want object or structural equality. The
		same problem is with hash function arguments. There's also a problem with possible memory leaks. You can
		pass a pointer to a string to the data structure and lose the pointer in the program, then when the
		structure deletes the pointer and the string stays in the memory to the program's end. Adding this to
		the framework would require passing the comparison function, hash function and some kind of destructor
		function to the data structure, or possibly use some reference counting system. It would be very
		technical and would be beyond the scope of this thesis. With an array or a big struct passed to the
		framework, there's a problem if we want to share the data structure and just copy the pointer, copying
		the whole thing may be a bad idea, because it can have quite a lot of data inside.

		\subsubsection{Linked data structures}

			When we want to store structs like \autoref{fig:linked-struct} in a data structure, the
			framework, at the moment, could generate some data structure for this kind of structs. The
			operation on the data structure would look like on \autoref{fig:struct-old}.

			\begin{figure}
				\lstinputlisting{thesis-pics/linked-struct.c}

				\caption{An example record, that we want to store in a data structure}

				\label{fig:linked-struct}
			\end{figure}

			\begin{figure}
				\lstinputlisting{thesis-pics/struct-old.c}

				\caption{Use of the dsinf calls, to perform some data structure operations with the
				records; not very comfortable to use}

				\label{fig:struct-old}
			\end{figure}

			In \autoref{fig:struct-old} a comparison function is used, it compares the heigth. We can use
			any function that compares a coordinate or a combination of coordinates, but only one order of
			the elements is available per data structure. This can also be achieved without using structs at
			all, like in \autoref{fig:struct-pointer-sizeof-reduction}.

			\begin{figure}
				\lstinputlisting{thesis-pics/struct-pointer-sizeof-reduction.c}

				\caption{Encoding of the problem of having one order data structure on records in C,
				into code using pointers}

				\label{fig:struct-pointer-sizeof-reduction}
			\end{figure}

			In the \autoref{fig:struct-pointer-sizeof-reduction} above we dereference structs as arrays of
			chars, to get to any field, we just take a pointer to a field in the array and cast it to the
			right type. So it doesn't really give us any more expressiveness. It would be nice if our
			program enabled things like comparing structs by more than one condition in one program. We
			would want to be able to use operations like in \autoref{fig:struct-new}.

			\begin{figure}
				\lstinputlisting{thesis-pics/struct-new.c}

				\caption{Examples of the new dsinf API, that takes different orders on structs into
				account}

				\label{fig:struct-new}
			\end{figure}

			For us to achieve such thing, we need to change how we choose structures for record types. First
			we modify the $DSU()$ definition for record variables.
			\todo{rest}

	\subsection{Upper bound on the element count}

		When analyzing the input program, we can try to get the information, if the maximum element count in the
		data structure is bounded by some constant. If we can obtain such an information (for most programs it
		probably will be undecidable), we can use it to enhance the data structure generated for the program.

		When the element count in the data structure is potentially infinite, we have to create an
		implementation of the structure, that allocates memory lazily, when it needs the space for new elements.
		We can imagine that a structure can allocate one chunk on every insertion and free a chunk on every
		deletion, or even allocate twice as big chunks and amortize the number of allocations to $O(lg n)$.

		When we have the information about the element count, we can allocate the whole static buffer for the
		data structure and this removes the whole cost of allocation during insertions and deletions.

		Using the technique described in \autoref{sec:transforming-on-line} we can transform our dynamic
		allocating data structure in a static one, if we are in a place in the program, from which we can't
		insert any more elements. This is an example of such program:

		\begin{figure}
			\lstinputlisting{thesis-pics/upper-bound-transform.c}

			\caption{An example showing a situation, where beneficial would be to transform a dynamic data
			structure into a similiar structure, but with all memory statically allocated already}

			\label{fig:upper-bound-transform}
		\end{figure}

		In \autoref{fig:upper-bound-transform} after the first loop, there are no more insertions into the data
		structure, so we can mark this spot and when we the control goes to that place, we transform the data
		structure into a static version of it, allocating a big chunk of memory and deallocating all that was
		left. This way to the end of the program we won't have to do any allocations.

	\subsection{Minimal element count threshold}

		It's worth noticing that we compare only the asymptotical complexity of data structures. Some awfully
		complicated structures can have good asymptotical results, but the constant is quite high. We can avoid
		this problem by setting a threshold for each structure, what is the smallest number of elements to use
		this data structure.

		Another problem arises, how to know at compile time, how many elements a data structure will have at
		runtime. We can ask the user to explicitly specify the number during compilation or we can try to
		detect how big the declared data is, with some kind of constant folding analysis, that checks if the
		insertion to the data structure is in a loop that is run more times than the threshold. This is a fragile
		solution, because not a lot of programs are easy to analyze this way.

		Another solution is runtime tallying the size, and using the method of transforming data structures
		on-line described in \autoref{sec:transforming-on-line}. When we hit a threshold for a data structure
		with a bigger constant, but overall better fitting to our task, we transform the old one into this one.

		\begin{figure}
			\lstinputlisting{thesis-pics/minimal-elem-count.c}

			\caption{An example showing a situation, where beneficial would be to transform a data structure
				into a more complex one, because the number of elements is sufficiently big, so the
				whole transformation is profitable}

			\label{fig:minimal-elem-count}
		\end{figure}

		In \autoref{fig:minimal-elem-count} \todo{what do we see}

\section{Program}

	The framework has a few modes of working:
	\begin{itemize}
		\item Recommendation mode - the framework analyzes the input files and the data structure operations in
			them, then outputs the recommended data structure on standard output
		\item Advice mode - the framework checks if any small change in the data structure operations would give
			the program a speed boost
		\item Compile mode - works as the Recommendation mode, but instead of printing the data structure to
			standard output, it compiles the files and links it with the chosen structure implementation.
	\end{itemize}
\end{document}
% vim: set wrap tw=120:
